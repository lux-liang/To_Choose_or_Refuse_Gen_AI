\documentclass[CTeX = true]{mcmthesis} 

\mcmsetup{
  tstyle=\color{black}\bfseries,
  tcn = 202600600,
  problem = F,
  sheet = true,
  titleinsheet = true,
  keywordsinsheet = false,
  titlepage = false,
  abstract = true
}

\usepackage{float}
\usepackage{txfonts}
\usepackage[utf8]{inputenc}
\usepackage{indentfirst}
\usepackage{graphicx}
\graphicspath{{picture/}}

\title{To Choose or Refuse Gen-AI:\\
A Data-Driven Analysis of Career Evolution and Educational Strategy}
\date{}

\begin{document}

% ===== Summary Sheet =====
\begin{abstract}
This paper examines how Generative Artificial Intelligence (Gen-AI) reshapes career trajectories in representative STEM, technical, and artistic occupations—software developers, electricians, and illustrators—and how higher education should respond. Using a task-based modeling framework, occupations are decomposed into weighted task units to quantify AI-driven substitution and augmentation effects, which are then aggregated to project employment dynamics and inform educational strategy. Results show that Gen-AI’s impact is governed by verification complexity: beyond a critical adoption threshold, congestion in human verification causes marginal AI benefits to turn negative, leading to employment decline. Targeted educational interventions can shift this threshold outward by strengthening graduates’ verification and oversight capabilities, thereby enhancing long-term employment resilience and guiding differentiated enrollment and curriculum design. 

\textbf{Keywords:} Task Decomposition; Verification Capacity; Employment Resilience


\end{abstract}


\maketitle

\tableofcontents
\newpage

% ---------- 1. Introduction ----------
\section{Introduction}
Generative Artificial Intelligence (Gen-AI) has rapidly evolved from a niche experimental tool into a pervasive force reshaping modern work environments. While its growing capabilities raise concerns about large-scale job displacement, evidence increasingly suggests that Gen-AI does not affect all occupations uniformly. Instead, its influence varies substantially across fields, depending on how work is structured and how human judgment interacts with automated systems.

A central challenge in evaluating Gen-AI lies in distinguishing between automation and augmentation. Many occupations are not eliminated outright; rather, they are transformed as routine tasks become automated while higher-level activities—such as verification, interpretation, and decision-making—remain human-centered. As a result, the impact of Gen-AI is better understood at the task level than at the occupation level alone.

This study focuses on three representative occupations—a software developer (STEM), an electrician (technical), and an illustrator (artistic)—to examine how Gen-AI reshapes career trajectories through task-level transformation. By decomposing each occupation into weighted task units, we assess how Gen-AI substitutes or augments specific tasks and how these effects aggregate to influence long-term employment trends under different adoption scenarios.

Beyond projecting employment outcomes, this paper addresses a core question posed by the problem: how should higher education institutions respond to Gen-AI? We develop a data-driven framework that links task-level AI impact to curriculum design and enrollment strategy, emphasizing the role of verification and oversight capabilities in sustaining employment resilience under increasing Gen-AI adoption.  Rather than framing the decision as choosing or rejecting Gen-AI, our analysis highlights the importance of selective integration aligned with occupational task structures.

The remainder of this paper is organized as follows. Section 2 introduces occupation selection and modeling assumptions. Section 3 describes data sources and preprocessing. Section 4 presents the task-based modeling framework and employment projection model. Section 5 analyzes results under multiple Gen-AI adoption scenarios. Section 6 provides targeted educational strategy recommendations, followed by broader evaluation considerations and concluding remarks.

\newpage
% ---------- 2. Occupation Selection and Assumptions  ----------
\section{Occupation Selection and Assumptions }

\subsection{Selected Occupations}

To capture the heterogeneous impacts of Generative Artificial Intelligence (Gen-AI) across the labor market, three representative occupations are selected from distinct categories: a STEM occupation (Software Developer), a technical occupation (Electrician), and an artistic occupation (Illustrator).

The software developer is chosen as a representative STEM occupation due to its strong exposure to Gen-AI tools in code generation, debugging, and documentation. This occupation involves a mix of routine coding tasks and high-level design and verification activities, making it well-suited for analyzing the balance between automation and augmentation.

The electrician represents technical occupations that rely heavily on physical execution, on-site problem solving, and safety-critical decision-making. While Gen-AI may assist in diagnostics, planning, and documentation, core tasks require human presence and manual skill, providing a contrast to purely digital professions.

The illustrator is selected to represent artistic occupations that emphasize creativity, visual interpretation, and authorship. Gen-AI has demonstrated strong capabilities in image generation and style imitation, directly affecting low-level production tasks while leaving conceptual design, narrative coherence, and ethical judgment largely human-centered.

Together, these three occupations exhibit clearly differentiated task structures, levels of physical dependency, and verification requirements, allowing for meaningful comparison of Gen-AI impacts and educational responses across occupational categories.

\subsection{Assumptions}

To ensure analytical tractability while preserving interpretability, the following assumptions are adopted throughout the modeling process:

\begin{enumerate}
    \item \textbf{Gradual Gen-AI Adoption.}  
    Gen-AI adoption is assumed to occur progressively rather than instantaneously. Adoption rates increase smoothly over time, allowing human workers and institutions to adapt through training and experience.

    \item \textbf{Task Decomposability.}  
    Each occupation can be decomposed into a finite set of representative tasks, and the relative importance of these tasks remains structurally stable over the modeling horizon.

    \item \textbf{Task-Level AI Impact.}  
    Gen-AI affects occupations through task-level substitution and augmentation rather than the direct elimination of entire occupations.

    \item \textbf{Educational Adaptation Effect.}  
    Educational curriculum design influences graduate adaptability and task alignment by enhancing verification, oversight, and higher-level cognitive skills, thereby indirectly affecting employment outcomes.In particular, education is assumed to enhance graduates’ capacity to verify, supervise, and contextualize AI-generated outputs. 

    \item \textbf{Comparative Interpretation.}  
    Model outputs are interpreted comparatively across occupations and scenarios rather than as precise long-term employment forecasts.
\end{enumerate}

\newpage
% ---------- 3. Model Development ----------
\section{Data Sources and Preprocessing }

This study integrates multiple publicly available and widely cited data sources to inform model calibration, task weighting, and scenario analysis. Data are selected to ensure transparency, reproducibility, and alignment with the task-level modeling framework.

\subsection{Employment and Labor Market Data}

Baseline employment levels, growth trends, and occupational outlooks are informed by labor market statistics published by authoritative institutions, including the U.S. Bureau of Labor Statistics (BLS), the Organization for Economic Co-operation and Development (OECD), and the World Economic Forum (WEF). These sources provide aggregate indicators of employment scale, historical growth rates, and projected demand under technological change.

Rather than relying on precise point forecasts, these data are used to establish relative demand trends and to calibrate comparative employment indices across occupations.

\subsection{Occupational Task Structure Data}

Task composition and task importance are derived from the Occupational Information Network (O*NET) database, which provides standardized descriptions of work activities, skills, and task relevance for a wide range of occupations.

For each selected occupation, representative tasks are identified and grouped into conceptual categories such as routine production, problem-solving, verification, and creative decision-making. Task weights are normalized to reflect their relative contribution to overall occupational output. These weights form the structural foundation of the task-level substitution and augmentation model.

picture

\subsection{Gen-AI Impact and Adoption References}

Estimates of Gen-AI capabilities, adoption trends, and task-level automation potential are informed by prior empirical and conceptual studies, including reports by the McKinsey Global Institute, Frey and Osborne (2017), and subsequent literature on artificial intelligence and labor markets.

These sources are used to guide parameter ranges and scenario design rather than to provide direct numerical inputs. In particular, they inform assumptions regarding which task categories are more susceptible to automation, which are more likely to be augmented, and how adoption rates may evolve over time.

\subsection{Preprocessing and Data Integration}

All data inputs are rescaled and normalized to ensure consistency across sources. Given differences in data granularity and temporal coverage, the model emphasizes relative comparisons and trend analysis over absolute prediction. This approach reduces sensitivity to measurement error while preserving interpretability and policy relevance.The roles of the data sources employed in this study are summarized in Table~\ref{tab:data_sources}.
\begin{table}
\centering
\caption{Data Sources and Their Roles in the Model}
\label{tab:data_sources}
\begin{tabular}{p{4cm} p{4cm} p{6cm}}
\toprule
\textbf{Data Source} & \textbf{Data Type} & \textbf{Role in Model} \\
\midrule
BLS / OECD & Employment statistics & Calibrate relative employment scale and trend \\
O*NET & Task descriptions and importance & Construct task decomposition and weights \\
McKinsey, Frey \& Osborne & AI impact assessments & Inform parameter ranges and scenario design \\
\bottomrule
\end{tabular}
\end{table}
\newpage
% ---------- 4. Model Development  ----------
\section{Model Development }

We develop a modular, task-based modeling framework that maps occupations to weighted task sets and evaluates the impact of Generative AI through interpretable task parameters. Formally, each occupation is represented as a finite set of tasks, where task weights reflect relative importance and time allocation. Task attributes are extracted from standardized occupational databases and serve as inputs to subsequent substitution and augmentation models.

By operating at the task level, the model captures heterogeneity within occupations and enables scenario-based projections of employment evolution under different Gen-AI adoption pathways. The resulting outputs provide a quantitative foundation for curriculum adjustment and institutional decision-making.
 
\subsection{Task Decomposition Model }

Each occupation is decomposed into a finite set of representative tasks $\{t_i\}_{i=1}^{n}$.
For each task, a weight $w_i$ is assigned to reflect its relative importance and time allocation within the occupation.

Task weights are constructed using O*NET task importance and frequency indicators and normalized such that
\begin{equation}
\sum_{i=1}^{n} w_i = 1 
\end{equation}
In addition to weights, each task is associated with a structured attribute vector $\mathbf{f}_i$, capturing interpretable characteristics such as
cognitive complexity, physical dependency, social interaction requirements,
and consequence of error.
These attributes serve as the foundation for modeling task-level AI substitution,augmentation, and verification requirements.

As shown in Figure~\ref{fig:occupation_task_decomposition}, each occupation is decomposed into weighted tasks that reflect relative importance and time allocation.
This decomposition is key to modeling task-level AI substitution and augmentation effects.
\begin{figure}
\centering
\includegraphics[width=0.55\textwidth]{occupation_task_decomposition}
\caption{Task Decomposition for Representative Occupations}
\label{fig:occupation_task_decomposition}
\end{figure}

This task decomposition structure allows for the quantification of task-level impacts and provides a clear framework for assessing how AI will augment or replace specific tasks within each occupation.

\subsection{Gen-AI Impact Index}

For each task $t_i$, we define a Gen-AI impact index $\Delta_i$ that measures
the net effect of AI adoption on task performance.
This index captures both productivity gains from automation or augmentation
and additional costs arising from verification, error correction, and oversight.

Formally, the impact index is defined as the difference between the expected
AI-assisted task cost and the human-only baseline:
\begin{equation}
\Delta_i = C_i^{H} - C_i^{AI},
\end{equation}
where $C_i^{H}$ denotes the cost of completing task $i$ using purely human labor,
and $C_i^{AI}$ represents the expected cost under AI assistance while ensuring
qualified delivery.

Positive values of $\Delta_i$ indicate net AI augmentation,
while negative values suggest that AI use becomes inefficient due to
verification burdens or error risks.Figure~\ref{fig:ai_impact_index} illustrates the Gen-AI impact index for different tasks in the selected occupations. Positive values of $\Delta_i$ indicate AI augmentation, while negative values suggest that AI use becomes inefficient due to verification and error correction requirements.
\begin{figure}
\centering
\includegraphics[width=0.75\textwidth]{ai_impact_index}
\caption{AI Impact Index for Different Tasks in Selected Occupations}
\label{fig:ai_impact_index}
\end{figure}

This graph visually represents how Gen-AI's impact varies across tasks, with certain tasks benefiting from automation and others being hindered by verification bottlenecks.

\subsection{Employment Projection Model}

Task-level impacts are aggregated to the occupational level using task weights.
The average net AI benefit for an occupation is defined as
\begin{equation}
\Delta_{\text{occ}} = \sum_{i=1}^{n} w_i \Delta_i.
\end{equation}

AI adoption at the occupational level is modeled as a bounded dynamic process,
where higher net benefits accelerate adoption, while verification congestion
imposes nonlinear penalties.
Employment evolution is then expressed as
\begin{equation}
E(t+1) = E(t)\left[1 + g_{\text{base}} + \eta \cdot G(A(t)) - \delta \right],
\end{equation}
where $A(t)$ denotes the effective AI adoption rate and $G(\cdot)$ captures
the net contribution of AI after accounting for verification capacity constraints.

When verification demand exceeds system capacity, congestion effects amplify
error-related costs, causing $G(A)$ to decrease with further adoption.
This mechanism generates a reversal threshold beyond which additional AI use
reduces employment resilience.

Figure~\ref{fig:employment_vs_ai_adoption} presents the projected employment trends as a function of Gen-AI adoption rates. The curve illustrates how employment levels fluctuate as AI adoption increases, with a notable reversal threshold beyond which further AI adoption leads to diminishing employment benefits.
\begin{figure}
\centering
\includegraphics[width=0.75\textwidth]{employment_vs_ai_adoption}
\caption{Projected Employment Trends vs. AI Adoption Rate}
\label{fig:employment_vs_ai_adoption}
\end{figure}


This figure captures the dynamic relationship between AI adoption and employment, illustrating the critical reversal threshold where the marginal benefits of AI use decrease due to verification capacity constraints.

\subsection{Education--Career Alignment Model}

Formally, educational intervention increases effective verification capacity,thereby shifting the functional response $G(A)$ upward for a given adoption level.
Educational interventions are incorporated by allowing curriculum design to modify key task parameters, particularly verification efficiency and oversight capacity.
Improved training reduces error probabilities, shortens verification time,and effectively expands system-level verification capacity.

As a result, educational strategies shift the reversal threshold to the right,enabling higher sustainable AI adoption while preserving positive employment outcomes.
This linkage provides a quantitative basis for evaluating curriculum reform,enrollment adjustment, and long-term workforce resilience.
As shown in Figure~\ref{fig:education_intervention_threshold}, educational interventions can shift the reversal threshold to the right, allowing for higher AI adoption without a decline in employment resilience. This shift occurs by enhancing the verification and oversight capacities of graduates.

\begin{figure}
\centering
\includegraphics[width=0.75\textwidth]{education_intervention_threshold}
\caption{Impact of Educational Intervention on the Reversal Threshold}
\label{fig:education_intervention_threshold}
\end{figure}


This chart illustrates the potential long-term benefits of integrating educational reforms focused on verification skills, which help delay the negative impact of AI adoption on employment.


\newpage
% ---------- 5. Results and Analysis   ----------
\section{Results and Analysis  }


\newpage
% ---------- 6. Educational Strategy Recommendations  ----------
\section{Educational Strategy Recommendations }

\newpage
% ---------- 7. Beyond Employment Metrics   ----------
\section{Beyond Employment Metrics }


Educational policy is not assumed to directly control labor demand, but rather to influence graduate adaptability and task alignment. By reducing skill mismatch and transition costs, curriculum design affects employment outcomes indirectly rather than deterministically.
\newpage

% ---------- 8. Strengths and Limitations ----------
\section{Strengths and Limitations}
\subsection{Strengths}
A primary strength of the proposed framework lies in its task-based structure. By decomposing occupations into constituent tasks and explicitly modeling automation and augmentation effects, the model captures heterogeneous Gen-AI impacts that aggregate employment-level approaches often overlook. This structure enhances interpretability and allows results to be directly linked to educational policy recommendations.
The model is also adaptable across occupations and institutions. Core parameters, such as task automation probability and augmentation coefficients, can be recalibrated using alternative datasets or expert assessments, enabling application beyond the three occupations studied. Furthermore, the scenario-based design supports robust comparison under varying Gen-AI adoption trajectories.

\subsection{Limitations}
Several limitations should be acknowledged. First, task composition and Gen-AI impact parameters rely partly on secondary data sources and stylized assumptions, which introduce uncertainty into long-term projections. While sensitivity analysis demonstrates that qualitative trends remain stable, precise quantitative forecasts should be interpreted with caution.
Second, the model assumes gradual and monotonic Gen-AI adoption, whereas real-world diffusion may be nonlinear and influenced by regulatory, economic, or societal factors not explicitly modeled. Additionally, feedback effects between education supply and labor demand are simplified, potentially understating dynamic market adjustments.
Despite these limitations, the model provides a structured and transparent framework for comparing relative impacts and informing strategic educational decisions in an evolving Gen-AI landscape.


\newpage
% ---------- 9. Conclusion    ----------
\section{Conclusion  }
While the model focuses on task-level impacts, its primary value lies in comparative analysis and policy guidance rather than precise prediction, aligning with the intended scope of educational decision-making.

\newpage
% ---------- 10. References     ----------
\section{References }
\newpage
\input{sections/11.AI Usage Report}

\end{document}
