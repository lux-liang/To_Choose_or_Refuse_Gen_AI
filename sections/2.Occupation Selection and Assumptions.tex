% ---------- 2. Occupation Selection and Assumptions  ----------
\section{Occupation Selection and Assumptions }

\subsection{Selected Occupations}

To capture the heterogeneous impacts of Generative Artificial Intelligence (Gen-AI) across the labor market, three representative occupations are selected from distinct categories: a STEM occupation (Software Developer), a technical occupation (Electrician), and an artistic occupation (Illustrator).

The software developer is chosen as a representative STEM occupation due to its strong exposure to Gen-AI tools in code generation, debugging, and documentation. This occupation involves a mix of routine coding tasks and high-level design and verification activities, making it well-suited for analyzing the balance between automation and augmentation.

The electrician represents technical occupations that rely heavily on physical execution, on-site problem solving, and safety-critical decision-making. While Gen-AI may assist in diagnostics, planning, and documentation, core tasks require human presence and manual skill, providing a contrast to purely digital professions.

The illustrator is selected to represent artistic occupations that emphasize creativity, visual interpretation, and authorship. Gen-AI has demonstrated strong capabilities in image generation and style imitation, directly affecting low-level production tasks while leaving conceptual design, narrative coherence, and ethical judgment largely human-centered.

Together, these three occupations exhibit clearly differentiated task structures, levels of physical dependency, and verification requirements, allowing for meaningful comparison of Gen-AI impacts and educational responses across occupational categories.

\subsection{Assumptions}

To ensure analytical tractability while preserving interpretability, the following assumptions are adopted throughout the modeling process:

\begin{enumerate}
    \item \textbf{Gradual Gen-AI Adoption.}  
    Gen-AI adoption is assumed to occur progressively rather than instantaneously. Adoption rates increase smoothly over time, allowing human workers and institutions to adapt through training and experience.

    \item \textbf{Task Decomposability.}  
    Each occupation can be decomposed into a finite set of representative tasks, and the relative importance of these tasks remains structurally stable over the modeling horizon.

    \item \textbf{Task-Level AI Impact.}  
    Gen-AI affects occupations through task-level substitution and augmentation rather than the direct elimination of entire occupations.

    \item \textbf{Educational Adaptation Effect.}  
    Educational curriculum design influences graduate adaptability and task alignment by enhancing verification, oversight, and higher-level cognitive skills, thereby indirectly affecting employment outcomes.In particular, education is assumed to enhance graduates’ capacity to verify, supervise, and contextualize AI-generated outputs. 

    \item \textbf{Comparative Interpretation.}  
    Model outputs are interpreted comparatively across occupations and scenarios rather than as precise long-term employment forecasts.
\end{enumerate}
