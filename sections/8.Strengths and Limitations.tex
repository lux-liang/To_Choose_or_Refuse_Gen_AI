
% ---------- 8. Strengths and Limitations ----------
\section{Strengths and Limitations}
\subsection{Strengths}
A primary strength of the proposed framework lies in its task-based structure. By decomposing occupations into constituent tasks and explicitly modeling automation and augmentation effects, the model captures heterogeneous Gen-AI impacts that aggregate employment-level approaches often overlook. This structure enhances interpretability and allows results to be directly linked to educational policy recommendations.
The model is also adaptable across occupations and institutions. Core parameters, such as task automation probability and augmentation coefficients, can be recalibrated using alternative datasets or expert assessments, enabling application beyond the three occupations studied. Furthermore, the scenario-based design supports robust comparison under varying Gen-AI adoption trajectories.

\subsection{Limitations}
Several limitations should be acknowledged. First, task composition and Gen-AI impact parameters rely partly on secondary data sources and stylized assumptions, which introduce uncertainty into long-term projections. While sensitivity analysis demonstrates that qualitative trends remain stable, precise quantitative forecasts should be interpreted with caution.
Second, the model assumes gradual and monotonic Gen-AI adoption, whereas real-world diffusion may be nonlinear and influenced by regulatory, economic, or societal factors not explicitly modeled. Additionally, feedback effects between education supply and labor demand are simplified, potentially understating dynamic market adjustments.
Despite these limitations, the model provides a structured and transparent framework for comparing relative impacts and informing strategic educational decisions in an evolving Gen-AI landscape.

