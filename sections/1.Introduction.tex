% ---------- 1. Introduction ----------
\section{Introduction}
Generative Artificial Intelligence (Gen-AI) has rapidly evolved from a niche experimental tool into a pervasive force reshaping modern work environments. While its growing capabilities raise concerns about large-scale job displacement, evidence increasingly suggests that Gen-AI does not affect all occupations uniformly. Instead, its influence varies substantially across fields, depending on how work is structured and how human judgment interacts with automated systems.

A central challenge in evaluating Gen-AI lies in distinguishing between automation and augmentation. Many occupations are not eliminated outright; rather, they are transformed as routine tasks become automated while higher-level activities—such as verification, interpretation, and decision-making—remain human-centered. As a result, the impact of Gen-AI is better understood at the task level than at the occupation level alone.

This study focuses on three representative occupations—a software developer (STEM), an electrician (technical), and an illustrator (artistic)—to examine how Gen-AI reshapes career trajectories through task-level transformation. By decomposing each occupation into weighted task units, we assess how Gen-AI substitutes or augments specific tasks and how these effects aggregate to influence long-term employment trends under different adoption scenarios.

Beyond projecting employment outcomes, this paper addresses a core question posed by the problem: how should higher education institutions respond to Gen-AI? We develop a data-driven framework that links task-level AI impact to curriculum design and enrollment strategy, emphasizing the role of verification and oversight capabilities in sustaining employment resilience under increasing Gen-AI adoption.  Rather than framing the decision as choosing or rejecting Gen-AI, our analysis highlights the importance of selective integration aligned with occupational task structures.

The remainder of this paper is organized as follows. Section 2 introduces occupation selection and modeling assumptions. Section 3 describes data sources and preprocessing. Section 4 presents the task-based modeling framework and employment projection model. Section 5 analyzes results under multiple Gen-AI adoption scenarios. Section 6 provides targeted educational strategy recommendations, followed by broader evaluation considerations and concluding remarks.
